\documentclass{article}
\usepackage[margin=1in]{geometry}
\usepackage{fancyhdr}
\usepackage{graphicx}
\usepackage{vhistory}
\usepackage[parfill]{parskip}
\graphicspath{{./}}

% Set fancy looking header/footer and move page number to the right
\pagestyle{fancy}
\fancyhead{}
\fancyfoot{}
\fancyfoot[R]{\thepage}

\title{}
\author{}
\date{}


%%%%%
% With communication of primary importance, the requirements specification should be in terms that meaningful for the
% problem domain. The reqspecs shoudl be organized in a readable format, concentrating on what the application is
% expected to acomplish. The reqspecs is a contract between the software development team and the customer so everyone
% is on the same page of what is to be developed an why.
%%%%%

\begin{document}
    \pagenumbering{gobble}

    % Titlepage, table of contents & list of figures on their own pages
    \begin{titlepage}
    \begin{center}
        \vspace*{1cm}

        \Huge
        \textbf{Title}

        \vspace{.5cm}
        \LARGE
        SubTitle

        \vspace{1cm}

        \textbf{authors}

        \vspace{.2cm}
        \Large
        01/01/1971

        \vspace{2cm}
        \includegraphics[scale=.35]{logo}

        \vfill

        Institution\\

    \end{center}
\end{titlepage}


    \tableofcontents
    \listoffigures

    % Revision history on its own page
    \newpage
    \begin{versionhistory}
        \vhEntry{0.1}{01.01.1971}{KT}{Document Creation}
    \end{versionhistory}
    \newpage

    % Start page numbering and the "meat" of the document
    \pagenumbering{arabic}
    \section{Introduction}
    %%
    % Purpose and intended audience of the document should be given and the layout of the remainder of the document
    %   should be discussed
    %%

    \section{Background}
    %%
    % Background material related to the problem domain should be provided. Examples are helpful in providing the
    %   technical information necessary to understand the problem to be addressed by the application.
    %%

    \section{Overview}
    %%
    % The purpose of the application should be provided, and the hardware and software environment for the development
    %   and installation of the application should be described. Users and how they will run the application should be
    %   discussed. The context within which the application will operate should be shown and described as well.
    %%

    \section{Environment}
    %%
    % The input and output devices and information repositories to be used by the application should be described in
    %   detail. Sample contents aid in communicating the function of information respositories within the application.
    %%

    \section{Operation}
    %%
    % Detailed information should be provided about the actions that will be performed by the application at the
    %   beginning of operation, during operation, and at the end of operation. Sample user interface screens are
    %   benificial in enabling understanding of the operational requirements for the application.
    %%

    \section*{References}
    %%
    % Any textbooks or other documents referenced in the requirements specification should be listed
    %%

    \appendix
    \section{Appendix}
    %%
    % Any additional material supporting the requirements specification should be provided.
    %%

\end{document}
